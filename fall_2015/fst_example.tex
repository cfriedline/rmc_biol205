\documentclass[12pt]{article}
\usepackage{fullpage}
\usepackage{amsmath}
\usepackage{hyperref}
\begin{document}
\pagestyle{empty}

\section*{$F_{ST}$ example
\footnote{Example borrowed from John Hawkes at \url{http://johnhawks.net}}}

Gene A is a key part of
the melanin expression pathway, which
contributes to
skin and hair pigmentation. A SNP that is strongly associated with lighter skin
pigment in Europe has two alleles, A and G, with G being associated with light
skin, at a frequency of 100\% in European-Americans. The SNP varies in frequency
in populations in the Americas with mixed African and American Indian ancestry. A sample in Mexico had 38\% A and 62\% G; in Puerto Rico the frequencies were 59\% A and 41\% G, and a sample of African-Americans from Charleston had 19\% A with 81\% G. What is the $F_{ST}$ in this example?

\vspace{1em}

To estimate $F_{ST}$, you can do the following:

\begin{itemize}
\item{Find the allele frequencies for each subpopulation}
\item{Find the average allele frequencies for the total population}
\item{Calculate the heterozygosity for each subpopulation}
\item{Calculate the average of the subpopulation heterozygosities.  This is
$H_S$}.
\item{Calculate the heterozygosity based on the total population allele
frequencies.  This is $H_T$}.
\item{Calculate $F_{ST}$}
\end{itemize}

\begin{equation*}
F_{ST} = \frac{H_T-H_S}{H_T}
\end{equation*}

\vspace{1em}
The data are below, for each population:

\begin{table}[h]
\centering
\begin{tabular}{lrr}
Population &  A & G \\
\hline
Mexican & 0.38 & 0.62 \\
Puerto Rico & 0.59 & 0.41 \\
African-American & 0.19 & 0.81 \\
\hline
\end{tabular}
\end{table}

\section*{Questions}
\begin{itemize}
\item{Given this value for $F_{ST}$, what can you say about the genetic similarity of
these populations?  Are they alike?  Very alike?  Somewhat dissimilar?  Very
dissimilar?}
\item{What can you say about the overall genetic diversity of these
populations?}
\item{We've previously defined $F_{ST}$ as $\frac{V_q}{\bar{p}(1-\bar{p})}$. 
Can you
think of how the formulation of $F_{ST}$ above is related to this?}
\end{itemize}
\end{document}